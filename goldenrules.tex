\documentclass[paper=a4]{scrartcl}
\usepackage{hyperref}
\usepackage{roboto}
\usepackage[utf8]{inputenc}
\sloppy
\frenchspacing
\usepackage[margin=2cm]{geometry}
\usepackage{scrlayer-scrpage}
\pagestyle{scrheadings}
\clearpairofpagestyles
\rohead{Version 2017-12-26 17:10}
\begin{document}
	\begin{center} \textsc{\Huge Golden Rules}\end{center}
	\hfill
	\section*{The Golden Rules for Speakers at the 34C3}
	\begin{itemize}
		\item Have fun and don't panic!
		\item \textbf{Please come to your lecture hall 20 minutes before your talk starts.} An
		angel will be there	for	you	to introduce you to	the stage equipment. Please, bring your own notebook.
		%If you did not bring a notebook for presenting, please talk to the Speakers' Desk so we can get you a speaker's notebook. Please let us know well ahead that you need one.
		\item All lectures are taped, streamed and will be distributed afterwards. Please repeat all questions and worthwhile comments from the audience which are not spoken into a microphone. Otherwise, they will not be on tape.
		\item Please make sure you don't extend your time slot, so the audience has a chance to change rooms.
		\item Leave some time for Q \&\ A after your talk!
		\item Once they are final, please hand in your slides at Speakers' Desk (CCL 9/10 on the 2nd floor) 
		\item[] OR upload them to the frab.cccv.de system directly.
		\item Translation: The talks are being translated to German. In order to help the interpreters please upload your slides to \url{https://speakers.c3lingo.org/}. If in doubt contact the interpreters via e-mail  to \url{lingo-content@lists.ccc.de} or on DECT 8726.
		\item Please switch off all tools (e.g. redshift) which alter the color temperatur according to ambient light or time of the day. Otherwise your beamer output will appear yellowish or reddish.
		\item We have document cameras available to show small objects on stage, like electronics, 3d-printed objkects or documents. In case you want to use them, please send an e-mail to \url{congress@c3voc.de} with name, location and time of the talk.
	\end{itemize}
	\section*{Frequently asked questions}
	\begin{itemize}
		\item \textbf{Can I drop my luggage somewhere?} Yes, you can store your luggage at the cloak room at the entrance and may take small pieces of hand luggage to the Speaker's Desk (we don't assume any liability, though). Make sure that you pick up the things you need before we leave for the night (around midnight, sometimes later but don't count on it).
		\item \textbf{Is it possible to check if the projector works with my notebook?} In most cases, you don't need to check that. Don't worry it will work. The aspect ratio of your slides should be 16:9 (1920x1080, 1600x900, 1280x720, \ldots). If you have really weird hardware talk to us (in time!) and we will work something out.
		\item We prefer digital connections such as DVI, HDMI and (mini-) Display-Port. VGA is only supported as a back-up solution. 
		\item \textbf{I'd like to work on my slides / prepare a few things. Do you have a quiet area for that?} Yes, we have a separate area where you can work in quiet at the Speakers' Desk.
		\item \textbf{There is something I'd like to know, I have a problem with \$Foo!} Come talk to us at the Speakers' Desk.
	\end{itemize}
\newpage
\begin{center} \textsc{\Huge Golden Rules}\end{center}
\section*{Goldene Regeln für die Vortragende (34C3-Edition)}
\begin{itemize}
	\item Have fun and don't panic!
	\item \textbf{Sei bitte 20 Minute vor Beginn deines Vortrags in deinem Vortragssaal.} Ein Engel wird dich dort in die Bühnentechnik einweisen. Solltest du kein Notebook mitgebracht haben, bitte jemanden darum, dir eins zu leihen.
	\item Alle Vorträge werden gefilmt, gestreamt und hinterher veröffentlicht. Bitte wiederhole daher jede Frage und jeden \textit{sinnvollen} Kommentar (aus dem Publikum/Internet), der nicht in ein Mikrofon gesprochen wurde für die Aufzeichnung.
	\item Bitte halte dich an die geplante Zeit deines Vortrags und plane am Ende deines Vortrags etwas Zeit für Fragen (und Antworten) ein.
	\item Sobald deine Vortragsslides fertig sind, reiche sie bitte am Speakers' Desk ein oder lade sie selbst ins frab.cccv.de
	\item Synchronübersetzung: Es hilft den Übersetzern deines Talks enorm, wenn sie sich vorab anhand deiner Slides über deinen Talk informieren können. Falls du unübliche (Fach-\nobreak) Wörter in deinem Talk benutzt, stelle diese bitte für die Synchronsprecher in einem Glossar zusammen. Bitte lade beides frühzeitig hoch über \url{https://speakers.c3lingo.org/}. Bei Fragen kontaktiere die Übersetzer über eine e-mail an \url{lingo-content@lists.ccc.de} oder unter DECT 8726.
	\item Um auf der Bühne kleine Objekte, wie Elektronik oder Dokumente, zeigen zu können, haben wir Dokumentenkameras zur Verfügung. Sofern du solch eine Kamera nutzen möchtest, sende bitte eine e-mail an \url{congress@c3voc.de} mit dem Name, Ort und Uhrzeit des Vortrages.
	\item Schalte Redshift und ähnliche Programme, die die Farbtemperatur der Tageszeit oder Beleuchtung anpassen ab. Sonst hat das Beamerbild leider einen Farbstich.
\end{itemize}
\section*{Frequently asked questions}
\begin{itemize}
	\item \textbf{Kann ich mein Gepäck hier lassen?} Du kannst dein Gepäck gegen eine Gebühr an der Garderobe abgeben. Kleines Handgepäck kannst du gerne am Speakers' Desk abstellen. Wir übernehmen keine Haftung. Bitte hole das Gepäck ab, bevor der Raum geschlossen wird. Der Speakers' Desk schließt gewöhnlich um Mitternacht.
	\item \textbf{Kann ich ausprobieren, ob mein Laptop mit dem Beamer funktioniert?} Ein Test sollte eigentlich nicht nötig sein, das wird schon gehen. Die Auflösung sollte im 16:9 Format sein (1920x1080, 1600x900, 1280x720 \ldots). Solltest du tatsächlich besonders außergewöhnliche Hardware anschließen wollen, rede (\textbf{rechtzeitig)}) mit uns, wir finden da einen Weg.
	\item Wir bevorzugen digitale Verbindungen wie DVI, HDMI oder (Mini-)Display-Port und unterstützen nur im Notfall VGA.
	\item \textbf{Ich brauche während meines Talks Netz, wo kann ich ausprobieren ob das funktioniert?} Wir haben im Speakers' Room genau das gleiche Netz wie in den Vortragssälen.
	\item \textbf{Wo kann ich noch ein paar Kleinigkeiten an meinen Folien ändern?}
	 Wir haben extra dafür eine ruhige Ecke im Speakers' Room (Raumnummer CCL 9/10 im zweiten Stock) vorgesehen.
	\item \textbf{Ich würde gerne noch etwas wissen, ich habe ein Problem mit \$Foo!} Rede mit uns am Speakers' Desk, wir versuchen dir zu helfen. Du kannst uns auf der 1020 oder 1021 anrufen (per DECT oder internem GSM))
\end{itemize}
\end{document}
